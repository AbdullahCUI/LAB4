\documentclass[12pt]{scrartcl}

\setlength{\parindent}{0pt}
\setlength{\parskip}{.25cm}

\usepackage{graphicx}

\usepackage{xcolor}

\definecolor{darkred}{rgb}{0.5,0,0}
\definecolor{darkgreen}{rgb}{0,0.5,0}
\usepackage{hyperref}
\hypersetup{
  letterpaper,
  colorlinks,
  linkcolor=red,
  citecolor=darkgreen,
  menucolor=darkred,
  urlcolor=blue,
  pdfpagemode=none,
  pdftitle={Lab 4.0 - Loops},
  pdfauthor={Christopher M. Bourke},
  pdfkeywords={}
}

\definecolor{MyDarkBlue}{rgb}{0,0.08,0.45}
\definecolor{MyDarkRed}{rgb}{0.45,0.08,0}
\definecolor{MyDarkGreen}{rgb}{0.08,0.45,0.08}

\definecolor{mintedBackground}{rgb}{0.95,0.95,0.95}
\definecolor{mintedInlineBackground}{rgb}{.90,.90,1}

%\usepackage{newfloat}
\usepackage[newfloat=true]{minted}
\setminted{mathescape,
               linenos,
               autogobble,
               frame=none,
               framesep=2mm,
               framerule=0.4pt,
               %label=foo,
               xleftmargin=2em,
               xrightmargin=0em,
               startinline=true,  %PHP only, allow it to omit the PHP Tags *** with this option, variables using dollar sign in comments are treated as latex math
               numbersep=10pt, %gap between line numbers and start of line
               style=default, %syntax highlighting style, default is "default"
               			    %gallery: http://help.farbox.com/pygments.html
			    	    %list available: pygmentize -L styles
               bgcolor=mintedBackground} %prevents breaking across pages
               
\setmintedinline{bgcolor={mintedBackground}}
\setminted[text]{bgcolor={mintedBackground},linenos=false,autogobble,xleftmargin=1em}
%\setminted[php]{bgcolor=mintedBackgroundPHP} %startinline=True}
\SetupFloatingEnvironment{listing}{name=Code Sample}
\SetupFloatingEnvironment{listing}{listname=List of Code Samples}


\title{CSCE 155 - C}
\subtitle{Lab 4.0 - Loops}
\author{~}
\date{~}

\begin{document}

\maketitle

\section*{Prior to Lab}

Before attending this lab:
\begin{enumerate}
  \item Read and familiarize yourself with this handout.
  \item Read the required chapters(s) of the textbook as 
  outlined in the course schedule.
\end{enumerate}

\section*{Peer Programming Pair-Up}

\textbf{For students in the online section:} you may complete
the lab on your own if you wish or you may team up with a partner
of your choosing, or, you may consult with a lab instructor to get
teamed up online (via Zoom).

\textbf{For students in the face-to-face section:} your
lab instructor will team you up with a partner.  

To encourage collaboration and a team environment, labs are be
structured in a \emph{peer programming} setup.  At the start of
each lab, you will be randomly paired up with another student 
(conflicts such as absences will be dealt with by the lab instructor).
One of you will be designated the \emph{driver} and the other
the \emph{navigator}.  

The navigator will be responsible for reading the instructions and
telling the driver what to do next.  The driver will be in charge of the
keyboard and workstation.  Both driver and navigator are responsible
for suggesting fixes and solutions together.  Neither the navigator
nor the driver is ``in charge.''  Beyond your immediate pairing, you
are encouraged to help and interact and with other pairs in the lab.

Each week you should alternate: if you were a driver last week, 
be a navigator next, etc.  Resolve any issues (you were both drivers
last week) within your pair.  Ask the lab instructor to resolve issues
only when you cannot come to a consensus.  

Because of the peer programming setup of labs, it is absolutely 
essential that you complete any pre-lab activities and familiarize
yourself with the handouts prior to coming to lab.  Failure to do
so will negatively impact your ability to collaborate and work with 
others which may mean that you will not be able to complete the
lab.  

\section{Lab Objectives \& Topics}
At the end of this lab you should be familiar with the following
\begin{itemize}
  \item How loop structures work and how to use them
  \item The difference between for-loops, while-loops, and do-while loops
  \item How to use loops to solve a problem
\end{itemize}

\section{Background}

Loop control structures allow us to design algorithms and procedures 
that repeatedly execute a block of code.  This allows us to repeatedly 
perform the same computation until some terminating condition is met, 
or to perform the same operation on a collection of data.  Three 
common loop control structures are the for-loop, the while-loop, and 
the do-while loop.  Though the syntax and behavior are slightly different, 
each one supports the same basic functionality.  In fact, loops are 
usually interchangeable: for example any for-loop can be rewritten 
as a while-loop and vice versa.

A for-loop has three basic components: an initialization statement, 
a terminating condition, and an iteration statement.  An example:

\begin{minted}{c}
int i;
for(i=0; i<10; i++) {
  printf("%d\n", i); 
}
\end{minted}

In this loop the first statement initializes the variable \mintinline{c}{i} 
to 0 while the third increments \mintinline{c}{i} by one on each 
iteration of the loop.  The second statement is the terminating 
condition: it is evaluated at the start of each iteration of the loop.  
If the condition is true, then the loop executes; if it evaluates to 
false, the loop does not execute any further.  The loop above 
would print out the values 0 thru 9.  On the 10th iteration, the 
variable i would have a value of 10 and thus \mintinline{c}{i < 10} 
would no longer be true and the loop ends.  Take careful note of the 
syntax and usage of semicolons.

A while-loop is similar, but does not collect all three components 
in the same line.  With a while loop, the terminating condition is 
still checked at the beginning of the loop.  Only if the condition is 
true does the loop execute.  

\begin{minted}{c}
int i=0;
while(i<10) {
  printf("%d\n", i);
  i++;
}
\end{minted}

In this example, the initialization is done outside, before the loop 
while the increment is done as part of the loop (inside the loop's 
code block).  

Another loop structure, the do-while loop, is slightly different in 
how it behaves.  Though it is similar to the while-loop, the key 
difference is that the do-while loop checks the terminating condition 
at the end of a loop's iteration.  As a consequence, the code block 
in a do-while loop is always executed at least once.

\begin{minted}{c}
int i;
do {
  printf("%d\n", i);
  i++;
} while(i<10);
\end{minted}

Note a slight syntax difference as well: there is a semicolon used 
at the end of the terminating condition. 

\section{Activities}

We have provided partially completed programs for each of the 
following activities.  Clone the lab's code from GitHub using the 
following URL: \url{https://github.com/cbourke/CSCE155-C-Lab04}.

\subsection{Computing Sine}

The standard math library provides a function to compute the sine 
of a given number.  Complex functions such as these are usually 
approximated using some numeric analysis technique.  One such 
technique is to use a Taylor series to approximate the sine function:

$$\sin{x} = \sum_{i=0}^\infty \frac{(-1)^i}{(2i+1)!} x^{2i+1} = x - \frac{x^3}{3!}+\frac{x^5}{5!}-\cdots$$

Obviously, we cannot compute an infinite series.  Instead, we will 
approximate $\sin(x)$ by computing the Taylor series above out to $n$ 
terms.

We have provided you with an incomplete program, \mintinline{text}{sine.c} 
that reads in two values from the command line, $x$ and $n$.  We have also 
written a function to compute the factorial which you may find useful.  
You will need to complete the program by writing code to compute 
the approximation of $\sin(x)$.

Complete the program and answer the questions on your worksheet. 

\subsection{Number Guessing Game}

Let's play a game.  In this game, a computer program randomly 
generates a number between 1 and 1,000.  You, the player, then 
make guesses as to what that number is.  If you guess correctly, 
the game ends and you win.  For each wrong guess, the computer 
program gives you a hint by telling you whether your guess is too 
low or too high.  The goal is to minimize the number of guesses 
you make.

We have provided an incomplete program, \mintinline{text}{guessingGame.c} 
that randomly generates a number between 1 and 1,000.  You 
need to write a loop structure that prompts the user for a guess.  
While that guess is wrong, the loop should continue.  It should 
also keep track of how many guesses the user makes and, for 
each wrong guess it should print the appropriate hint to the user.

Complete the program and answer the questions on your worksheet. 

\subsection{Nesting Loops}

It is possible (and often necessary) to \emph{nest} loops.  That is, to 
perform a loop within another loop (just as you can nest conditional 
statements).  The syntax is straightforward: you simply write another 
loop within a loop.  For each iteration that the ``outer'' loop executes, 
the ``inner'' loop goes through its entire execution.  As an example, 
consider the following code snippet with corresponding output.

\begin{minted}{c}
  int i, j;
  for(i=0; i<3; i++) {
    for(j=1; j<5; j++) {
      printf("%d ", (i+j));
    }
    printf("\n");
  }
\end{minted}

Output:

\begin{minted}{text}
1 2 3 4
2 3 4 5
3 4 5 6
\end{minted}

\subsubsection*{Primality Testing}

Recall that an integer is \emph{prime} (ex: 2, 3, 5, 7, etc.) if it 
is only divisible by 1 and itself.  Otherwise, it is called composite.  
Consider the following algorithm to determine if an integer $x$ is 
prime: for each integer $y$ from 2 up to the square root of $x$, if $y$ 
divides $x$, then $x$ is composite (not prime).  Otherwise if no such 
integer in that range divides $x$, then $x$ is prime.  

Starter code has been provided in \mintinline{text}{primes.c}
that reads in an integer $n$ as a command line argument.  Then, for 
each integer 2 up to $n$, it outputs the integer 
and whether it is prime or composite.  You will need to complete
the program by writing a nested loop that determines if a value
is prime or composite.  Demonstrate your working 
program to a lab instructor and have them sign your worksheet.

\section{Handin/Grader Instructions}

\begin{enumerate}
  \item Hand in your completed files:
  \begin{itemize}
    \item \mintinline{text}{sine.c}
    \item \mintinline{text}{guessingGame.c}
    \item \mintinline{text}{primes.c}
    \item \mintinline{text}{worksheet.md}
  \end{itemize}
  through the webhandin (\url{https://cse-apps.unl.edu/handin}) 
  using your cse login and password.  
  \item Even if you worked with a partner, you \emph{both} should
  turn in all files.
  \item Verify your program by grading yourself through the
  webgrader (\url{https://cse.unl.edu/~cse155e/grade/}) using the
  same credentials.
  \item Recall that both expected output and your program's output
  will be displayed.  The formatting may differ slightly which is fine.
  As long as your program successfully compiles, runs and outputs 
  the \emph{same values}, it is considered correct.
\end{enumerate}


\section{Advanced Activities (Optional)}

Write a program that is the ``reverse'' of the number guessing game.
That is, the computer is the player and the user is the one who
picks a number between a certain range.  The computer should
then formulate an optimal guessing strategy and the user indicates
if the actual number is correct, lower or higher.  The program
should be ``smart enough'' to know when the user is cheating.

\end{document}
